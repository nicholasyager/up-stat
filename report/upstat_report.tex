\documentclass{report}

% PACKAGES
\usepackage[margin=1.00in]{geometry}
\usepackage[pdftex]{graphicx}
\usepackage[table,xcdraw]{xcolor}
\usepackage{indentfirst}
\usepackage{caption}
\usepackage{float}
\usepackage{caption}
\usepackage{subcaption}

% \usepackage{natbib}

\bibliographystyle{ieeetr}

\usepackage{Sweave}
\begin{document}
\Sconcordance{concordance:upstat_report.tex:upstat_report.Rnw:%
1 7 1 1 0 30 1}

\begin{titlepage}

\begin{center}

~\\[4cm]

\textsc{\Large Culver Road and East Main Street Intersection}\\[1.5cm]
\textsc{\huge Traffic Analysis Report}\\[8.5cm]



\vfill

{ March 20, 2015}

\end{center}
\end{titlepage}

\noindent
\section*{Introduction}

%This subsection should include a diagram of the interesction and some statistics
%about the interection. This should include information about the traffic flow
%rate (volume), the LOC classification, and some general information that is
%useful to the DOT.

There has been intense study in the area of traffic management since the mid-1990s. The ability to quantify and discern underlying patterns by use of a simple loop detector is a useful tool in understanding the complex nature of transportation systems. We were tasked with identifying unique trends in the data and to provide sound recommendations to improve traffic conditions for the intersection of Culver Road and East Main Street in Rochester, New York. The data, obtained from a loop detector approximately 200 feet away from intersection in the southbound lane, has nearly eight months of data stretching from October of 2013 to June of 2014. The loop detector transmits data in five minute intervals, and calculates pertinent metrics such as estimated hourly volume, number of cars approaching a red light, delay, and others.

The intersection, pictured in \ref{fig:intersection}, is part of an important byway between Bay Street and East Main Street. Zoning data provided by the city of Rochester shows us that the area surrounding this portion of Culver Road is mainly residential. However, there are designated commercial zones at both the intersections of Bay and East Main, and additional commercial lots in between both of these locations \cite{PropInfo}. These are mainly smaller business and not large scale commercial plazas similar to those in Henrietta or elsewhere in the city of Rochester. It is important to consider this information in conjunction with the results of our analyses when providing meaningful suggestions.

We also consider the Traffic Volume Map that is provided by the County of Monroe and the City of Rochester. Culver Road normally experiences a large amount of traffic that is comparable to East Main \cite{TrafficMap}. Furthermore, it appears as though the traffic continues down Culver and that there is no noticeable amount of traffic turning onto East Main that would contribute to any possible congestion.

\begin{figure}[H]
        \centering
        \begin{subfigure}[b]{0.45\textwidth}
                \includegraphics[width=\textwidth]{satellite_H.png}
        \end{subfigure}
        \begin{subfigure}[b]{0.45\textwidth}
                \includegraphics[width=\textwidth]{streetview.png}
        \end{subfigure}
        \caption{Pictured Left: satellite image of the intersection of Culver and East Main. Culver Road is highlighted in a light green. To the bottom right is East High School of the Rochester School District. Pictured Right: A streetview of Culver Road as it approaches East Main. Several businesses are located on both sides of the street. Location is approximate to where the loop detector is installed. Both images courtesy of Google Maps.}
        \label{fig:intersection}
\end{figure}

\noindent
\section*{Data Analysis}

To assess the traffic conditions at the Culver Road and East Main Street
intersection, we determined the current state of the intersection in terms of
federal DOT guidelines for this class of intersection, as well as analyses to
examine trends and patterns in the traffic recorded and to predict occurances of
traffic congestion based on time of day, day of the week, and the weather
conditions present.


\subsection*{Traffic Analysis}

There are a few common characteristics to examine when determining the traffic
flow for a particular intersection. Namely, it is important to determine the
amount of delay experience by drivers as they enter the intersection, the density
of the vehicles as they pass through the intersection, and the velocity at which
traffic flows through the intersection. With these three variables, it is possible
to determine the effects of traffic congestions on the flow of traffic through
the intersection.

The amount of time that vehicles wait at an intersection is refered to as the
Level of Service (LOS). Ranked in letter grades from A to F, the LOS is an
identifier for the overal health of the intersection. Ideally, an intersecton
should be classified as being either A, B, or C, denoting free flow, reasonably
free flow, and stable flow, respectively. The 2010 Highway Capacity
Manual classifications for LOS can be seen in Table \ref{LOStable}, in which grade
A intersections have less then 10 seconds of vehicle control delay, whereas grade
F intersections have more than 80 of vehicle control delay.

\begin{table}[h]
\centering
\caption{Level of Service classifications published in the 2010 Highway Capacity
Manual.}
\begin{tabular}{c | c}
\textbf{LOS} & \textbf{Vehicle Control Delay (Sec.)}\\\hline
A & $\le 10$\\
B & $10 - 20 $\\
C & $20 - 35$\\
D & $35 - 55$\\
E & $55 - 80$\\
F & $\ge 80$\\
\end{tabular}
\label{LOStable}
\end{table}

Using these classifications, we examined the average vehicle control delay for
each five minute observation period. As shown in Figure \ref{fig:LOSfigure},
the majority of observations have a Grade F Level of Service (58.38\%). Each of
the other levels of service occured in between 7.45\% and 8.63\% of observations.
This suggests that the Culver Road and East Main Street intersections is
experienceing traffic congestion, in which each vehicle move in lock step with
the vehicle in front of it \cite{HCM}.

\begin{figure}[h]
\centering
\includegraphics{upstat_report-LOCplot}
\caption{The Culver Road/East Main Street
intersection has a poor Level of Service. The histogram on the left shows the
distribution of the average vehicle control delay for each observation. The
distribution is bimodal with peaks at zero seconds and 140 seconds. The median
delay was 103 seconds with a IQR of
123 seconds. The bar plot on the right shows the
distribution of each LOS grade for each observation in our data set.
Observations were given a letter grade based on the average vehicle control
delay for each 5 minute interval. The majority of all observations had a delay
of greater than 80 seconds, suggesting that the intersection is predominantly grade
F.}
\label{fig:LOSfigure}
\end{figure}

We also examined the relationship between the traffic density and traffic volume,
also known as flux. Whe plotted, the relationship between the traffic density
and the traffic volume creates the fundamental diagram of traffic flow
\cite{HCM}, as shown in Figure \ref{fig:Fundamental}. Using linear regression techniques, we
were able to estimate the free flow velocity to be 19.5 miles per hour and the
traffic wave velocity to be 2.43 miles per hour against the direction of traffic.
We also determiend thecritical traffic density to be 42.8 vehicles per mile.
As the vehicle denisty passes the critical density, the traffic flow becomes
more unstable leading to traffic waves and congestion. To maximize
the traffic flow, the vehicle density must remain below the critical density.

\begin{figure}[h]
\centering
\includegraphics{upstat_report-003}
\caption{Fundamental diagram of traffic flow provides
free flow and traffic wave velocities. Traffic density was calculated by
dividing the traffic volume by the traffic speed. Using linear regression, we
found the free flow velocity (19.5 mph) as depicted in red, the traffic wave
velocity (2.43 mph) as depicted in blue, and the critical density (42.8 vehicles
per mile) as depicted in green.}
\label{fig:Fundamental}
\end{figure}


\subsection*{Trend Analysis}

To examine the general trend in the metrics we were provided, we performed linear
and multivariate regression. These techniques allows us to approximate the traffic
volume and delay. Figure \ref{trends} shows the median delay for each week and
the median traffic volume for each week with the linear regression for each data
set. Our data suggest that the median amount of time that vehicles wait at the
intersection may be increasing by 0.24 seconds per week. Our data also suggest
that the median number of vehicles utilizing the intersection may be increasing
by approximately 11 vehicles per week. Although the correlation coefficients are
rather low, we are confident that the traffic congestions will worsen as more
vehicles utilize the intersection.

\begin{figure}[H]
\centering

\includegraphics{upstat_report-VolumePlot}
\caption{The median traffic delay and median weekly
traffic volume are increasing over time. A linear regression of the median
traffic delay for each week suggests that the traffic delay may be increasing
0.24 seconds per week
($R^2 = 0.13$). Similarly, a linear
regression of the median number of vehicles to pass through the intersection per
week suggests that the number of vehicles utilizing the intersection may be
increasing at a rate of 11.08 vehicles
per week ($R^2 = 0.24$).}
\label{trends}
\end{figure}

\subsection*{Pattern Analyses}

A secondary way of understanding the traffic passing though the Culver Road and
East Main Street intersection is to assess the prevailing patterns of traffic.
Two methods of analyzing traffic patrerns are clustering and correlation analysis.
Clustering algorithims give us an opportunity to treat each day's traffic like
a single data point. By comparing the distance between each day's traffic, we can
group similar traffic patterns together. Correlation analyses, on the other hand,
examine how similar each individual pairing of days is by comparing each traffic
observation together. These two different techniques provide different ways of
finding patterns: clustering allows us to find similar large-scale similarities
between days, whereas correlations gives a more granular measure of similarity
between a day and the expected traffic.

Using a CLARA clustering algorithim, we clustered the daily traffic volume
for each day of data using 10-minute intervals. We found that there are three
primary types of traffic: weekday traffic, Saturday traffic, and Sunday traffic.
Each of these types in named after the days in which they are most likely to
occur. Figure \ref{fig:clustering}A, shows a cluster plot of the three clusters
of traffic measured along their principle components. The circles represent
traffic classified as Sunday traffic, the triangles are weekday traffic, and
the pluses are Satuday traffic. It should be noted that the most variable traffic
yupe is the Sunday classification, suggesting that there may be incidences of
highly irregular traffic that are more similar to Sundays than any other day
of the week. A notable example of this is Thanksgiving Thursday on November 28,
2013. Thanksgiving experiences durastically lower traffic than both the typical
Sunday and the typical Thursday, but the low traffic volume is more similar to
a Sunday so it was clustered accordingly, as shown in Figure \ref{fig:clustering}B.

\begin{figure}[H]
\centering
\begin{subfigure}[b]{0.45\textwidth}
\centering
A\\
\includegraphics{upstat_report-clustering}
\end{subfigure}
\begin{subfigure}[b]{0.45\textwidth}
\centering
B\\
\includegraphics{upstat_report-007}
\end{subfigure}
\caption{CLARA clustering suggests three primary traffic types.}
\label{fig:clustering}
\end{figure}


For our correlation analysies we compared each day's traffic volume to the median
traffic for that day of the week. For example, we examined the correlation between
the traffic volume for Thanksgiving to the median traffic volume at each 5-minute
interval for all Thusdays. A plot of each day's correlation, as shown in Figure \ref{fig:correlation},
outlines the variablity in each day's traffic patterns. It should be noted that
the days of abnormally low correlation relate to weekdays clustered as Sunday
traffic in Figure \ref{fig:clustering}B. In this figure high correlation suggests that
the day's traffic is very similar to the expected traffic for that day of the week,
whereas a low correlation is indicative of an abnormality in traffic volume.
One interesting case of abnormal traffic is the seventh Thursday of the data set,
which corresponds to Thanksgiving. The 2013 Thanksgiving traffic was abnormally
low for a Thursday. Similarly, the 10th and 20th Wednesdays of the data set
exhibit similar patterns of minimal traffic. Interestingly, weekdays have a higher
median correlation (0.946) and a lower
variance in correlation (0.003) than the weekends. This
suggests that weekdays have a relatively stable or predictable traffic pattern
with a few dramatic exceptions.

\begin{figure}[h]
\centering
\includegraphics{upstat_report-008}
\caption{Correlation analysis shows holidays, data collection errors, and small-scale
traffic patterns. In this color scale, white indicates high correlation, purple
indicates moderate correlation, and orange indicates low correlation. Some days of
interest include the seventh Thursday, which corresponds with Thanksgiving, the
10th Tuesday and Wednesday, and the 20th Wednesday.}
\label{fig:correlation}
\end{figure}


\subsection*{Bayesian Analysis}

In order to determine how the probability of congestion relates to the time of day we adapted Bayes' Theorum for use with the data provided. Because the traffic on the weekend is not as intense as the traffic during the weekdays, the decision was made to evaluate each day independently of one another so as not to induce any sort of bias into the results. Congestion was determined simply by using the critical density of the intersection, identified as 42.8 Vehicles per Mile. Any situation where the density is equal to or greater than this measure is determined to be 'congested.' Any situation below this measure is simply 'not congested.'

This is then sorted for each time step, and a proportion is generated about how many times (on a specific day of the week) that particular time is considered congested. For example, this means that if 2 out of the 10 observations at 2:10am on Wednesdays is considered to be congested, equal or exceeding the critical density, the proportion is calculated to be 0.2. This is then multiplied by the probability of congestion at any time amongst all the observations, and is then divided by the probability of randomly selecting that time from the dataset.

Figure \ref{fig:bayesplot} provides a way of seeing the probability of congestion
occuring given the day of the week and the time of day. As expected the probability
of congestion increases dramatically between 4 and 6 PM, which correspond with the
traditional rush hour period. Interestingly, there is a non-trivial probability of
traffic congestion occuring between 2 and 6 AM. This may correspond with people
who work during the evening. Additionally, there is a slight spike at 7 AM, which
may correspond with school traffic or people leaving for jobs with 7:20 or 8 AM
starting times.

\begin{figure}[h] \label{fig:bayesplot}
\centering
\includegraphics{upstat_report-bayesplot}
\caption{Probability of observing congestion based on specific times throughout the day (represented as p(C|T)). All days are represented as separate lines. A maximum probability of 91.2\% is shown on the graph and occurs at 5:15pm on Friday. Most weekdays have the highest probability of congestion between 5:00pm and 5:25pm, or during evening rush hour.}
\end{figure}

Table \label{basestable} presents times that have he highest probability of observing traffic congestion. For the purposes of this table Saturday and Sunday were omitted from the table due to the fact that their relative probabilities were approximately zero. A point of interest
is that congested periods start earlier on Fridays, suggesting that people may
be leaving work early on Fridays.

\begin{table}[h] \label{bayestable}
\centering
\caption{Selected results from Bayesian analysis of traffic data.}
\begin{tabular}{l|l
>{\columncolor[HTML]{EFEFEF}}l l
>{\columncolor[HTML]{EFEFEF}}l l}
\textbf{Time} & \textbf{Mon} & \textbf{Tues}                & \textbf{Wed} & \textbf{Thur} & \textbf{Fri}                                         \\ \hline
17:05         & 0.500          & 0.471                        & 0.529        & 0.588         & \cellcolor[HTML]{FFFFFF}0.500                          \\
17:10         & 0.765        & 0.824                        & 0.677        & 0.824         & \cellcolor[HTML]{FFFFFF}{\color[HTML]{333333} 0.853} \\
17:15         & 0.824        & {\color[HTML]{333333} 0.853} & 0.882        & 0.824         & 0.912                                                \\
17:20         & 0.853        & 0.882                        & 0.765        & 0.882         & 0.677                                                \\
17:25         & 0.588        & 0.794                        & 0.647        & 0.677         & 0.529
\end{tabular}
\end{table}

\subsection*{Weather Analysis}
Praesent iaculis pulvinar justo eu cursus. Mauris a ex eget sapien finibus commodo. Aenean et malesuada nunc. Nam iaculis porttitor faucibus. Etiam consequat rhoncus eros ut vehicula. Mauris vel ligula non nisl posuere aliquet. Mauris ac pulvinar metus, feugiat faucibus metus. Proin vel gravida tellus. Phasellus quis metus sollicitudin, volutpat massa nec, euismod risus. Nunc iaculis pellentesque purus non lobortis. Nam ac neque varius, auctor velit in, pretium elit. Ut facilisis, lacus nec mattis aliquet, nisi sapien vehicula massa, nec pharetra velit orci ac magna. Pellentesque vel velit ornare, aliquam turpis sed, egestas lacus. Aliquam dapibus commodo nulla, nec tincidunt eros lacinia ut. Maecenas mi tellus, cursus at neque vitae, consectetur tempus eros.

Aenean maximus metus metus, in varius lorem ultrices in. Nulla a tortor tempus, facilisis orci in, viverra augue. Maecenas viverra mattis velit, et volutpat elit finibus eu. Curabitur eu diam sed justo malesuada fringilla. Fusce ac nisl nec diam aliquam lobortis. Fusce fermentum ligula dui, vel condimentum erat convallis vitae. Etiam finibus dolor et feugiat iaculis. Duis varius nulla non dictum rutrum. Curabitur aliquet turpis id turpis fringilla, laoreet interdum mauris dictum. Praesent id pharetra tortor, ac laoreet urna. Sed porta lacus non sagittis condimentum.
\noindent\section*{Recommendations}
Praesent iaculis pulvinar justo eu cursus. Mauris a ex eget sapien finibus commodo. Aenean et malesuada nunc. Nam iaculis porttitor faucibus. Etiam consequat rhoncus eros ut vehicula. Mauris vel ligula non nisl posuere aliquet. Mauris ac pulvinar metus, feugiat faucibus metus. Proin vel gravida tellus. Phasellus quis metus sollicitudin, volutpat massa nec, euismod risus. Nunc iaculis pellentesque purus non lobortis. Nam ac neque varius, auctor velit in, pretium elit. Ut facilisis, lacus nec mattis aliquet, nisi sapien vehicula massa, nec pharetra velit orci ac magna. Pellentesque vel velit ornare, aliquam turpis sed, egestas lacus. Aliquam dapibus commodo nulla, nec tincidunt eros lacinia ut. Maecenas mi tellus, cursus at neque vitae, consectetur tempus eros.

Aenean maximus metus metus, in varius lorem ultrices in. Nulla a tortor tempus, facilisis orci in, viverra augue. Maecenas viverra mattis velit, et volutpat elit finibus eu. Curabitur eu diam sed justo malesuada fringilla. Fusce ac nisl nec diam aliquam lobortis. Fusce fermentum ligula dui, vel condimentum erat convallis vitae. Etiam finibus dolor et feugiat iaculis. Duis varius nulla non dictum rutrum. Curabitur aliquet turpis id turpis fringilla, laoreet interdum mauris dictum. Praesent id pharetra tortor, ac laoreet urna. Sed porta lacus non sagittis condimentum.
Nuke Rochester.

\begin{thebibliography}{9}
  \bibitem{PropInfo}
    City of Rochester,
    \emph{Property Information Application}.
    http://maps.cityofrochester.gov/propinfo/.
  \bibitem{TrafficMap}
    City of Rochester,
    \emph{Traffic Volume Map}.
    2014.
    http://www2.monroecounty.gov/files/dot/pdfs/City-adt-map-through-2014.pdf.
  \bibitem{HCM}
    Transportation Research Board of the National Academies,
    \emph{Highway Capacity Manual 2010}.
    5th edition,
    2010.
\end{thebibliography}

\end{document}
